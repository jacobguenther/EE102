\documentclass{article}
\usepackage{caption}
\usepackage{amssymb}
\usepackage{array}
\usepackage{geometry}
\usepackage{scrextend}
\usepackage{amsmath}
\usepackage{hyperref}
\usepackage{graphicx}
\usepackage{pdfpages}
\usepackage{multicol}
\usepackage{tabularx}
\usepackage{float}

\title{EE102 Homework 2}
\author{Jacob Guenther}

\geometry{
	a4paper,
	total={170mm,257mm},
	left=20mm,
	top=20mm,
}

\begin{document}

\includepdf[pages=1,pagecommand={}]{Lab_2_cover.pdf}

\section{Objective}
The goal of this lab is to gain more experience using a multimeter while exploring some of the properties of a linear voltage regulator. In it we use a multimeter to measure node and differential voltage. Then we break the circuit to measure current. Finally we compare our measured results to a simulation of the circuit.

\section{Equipment}
\begin{itemize}
	\item Agilent 34410A Multimeter
	\item Agilent E354xA Dual Output Power Supply
	\item Prototyping Board
	\item Linear Voltage Regulator
	\item Resistors
	\item Capacitor
	\item Jumpers
\end{itemize}

\section{Setup}
The circuit used in this lab is shown in figure 1.

\section{Observations and Results}

\begin{table}[H]
\begin{tabularx}{\textwidth}{ | X | X | X | X | X | X | }
	\hline
	\textbf{Node Voltage $\text{V}_A$} &
	\textbf{Node Voltage $\text{V}_B$} &
	\textbf{Differential Voltage $\text{V}_{AB}$} &
	\textbf{Current Through 1 k$\Omega$} &
	\textbf{Differential Voltage Calculated $\text{V}_A-\text{V}_B$} &
	\textbf{Calculated Resistance} \\
	\hline
	1.5 & 0.00020238 & & & & \\
	2.0 & 0.669      & & & & \\
	2.5 & 1.189      & & & & \\
	3.0 & 1.611      & & & & \\
	3.5 & 2.096      & & & & \\
	4.0 & 2.583      & & & & \\
	4.5 & 3.072      & & & & \\
	5.0 & 3.563      & & & & \\
	5.5 & 4.0563     & & & & \\
	6.0 & 4.53       & & & & \\
	6.5 & 4.924      & & & & \\
	7.0 & 5.067      & & & & \\
	7.5 & 5.069      & & & & \\
	8.0 & 5.07       & & & & \\
	8.5 & 5.066      & & & & \\
	9.0 & 5.069      & & & & \\
	\hline
\end{tabularx}
\caption{\label{tab:table-name}Displays the measured node and differential voltages, and current, as well as calculated differential voltage and resistance.}
\end{table}

\newpage
\section{Conclusion}


\subsection{Sources of Error}

\begin{itemize}
	\item 
\end{itemize}


\newpage
\section{References}
\noindent
[1] Denise Thorsen, Maher Al-Badri, INTRODUCTION TO ELECTRICAL AND COMPUTER ENGINEERING, University of Alaska Fairbanks, 2022.
\newline
\newline
\noindent

\end{document}